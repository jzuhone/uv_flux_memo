\documentclass[11pt]{article}


\newcommand{\putdraft}{\special {!userdict begin /bop-hook{gsave 200 30
translate 65 rotate /Times-Roman findfont 216 scalefont setfont 0 0
moveto 0.9 setgray (DRAFT) show grestore}def end}}

\def\quote{\list{}{\rightmargin\leftmargin}\item[]}
\let\endquote=\endlist

\def\changemargin#1#2{\list{}{\rightmargin#2\leftmargin#1}\item[]}
\let\endchangemargin=\endlist

%%% use the geometry package to specify the margins.  up the
%%% headheight a bit to give more room for the headers
\usepackage[head=12pt]{geometry}
\usepackage{graphicx}
\usepackage{amsmath}
%%% the possible options to the memo package are:
%%%
%%%  NofM - output pagenumbers as N / M, where M is the last page.
%%%         This requires two LaTeX runs.  Only works if fancy is specified
%%%
%%%  prerule - draw a rule between the memo preamble and the memo body
%%%
%%%  fancy - use the fancyhdr package to make nice page headers and footers
%%%          if you don't like the default, go ahead and override the
%%%          fancyhdr stuff directly.  The default handles twosided
%%%          documents.
%%%
%%%  rcs - use the rcsinfo package to generate the contents of the Date,
%%%        Version, and File memo header fields.  This is very useful
%%%        if you are using the RCS package to manage your text.
%%%        You will have to put the following text in your file:
%%%
%%%             \rcsInfo $Id$
%%%
%%%        It will be updated by RCS when you commit your changes.
%%%        If you don't specify rcs, the above line will cause LaTeX
%%%        to complain. You will get all three fields with this option.
%%%        You can override the contents of the fields by specifying
%%%        them explicitly. To delete them, use \DeleteMemoField, e.g.:
%%%
%%%             \DeleteMemoField{File}
%%%
\def\tablenotemark#1{\rlap{$^{\rm #1}$}}
%%% This template shows off all of the options
%%% If you're not using RCS, remove the rcs option.
\usepackage[prerule]{memox_ppp}

%%% grab the CXC letterhead, specifying the CfA address
\usepackage[addr=cfa]{cxc_letterhead}
%%% and tell the memo package to use it
\memoletterhead{\CXCletterhead}

\begin{document}

\newcommand{\chandra}{{\em Chandra}~}
\newcommand{\rosat}{{\em ROSAT}}
\newcommand{\chase}{{\em ChASeM33}}
\newcommand{\etal}{{\em et al.}}
\newcommand{\be}{\begin{enumerate}}
\newcommand{\ee}{\end{enumerate}}
\newcommand{\bc}{\begin{center}}
\newcommand{\ec}{\end{center}}
\newcommand{\bi}{\begin{itemize}}
\newcommand{\ei}{\end{itemize}}
\newcommand{\bd}{\begin{description}}
\newcommand{\ed}{\end{description}}
\newcommand{\bt}{\begin{tabbing}}
\newcommand{\et}{\end{tabbing}}
\newcommand{\code}[1]{\texttt{#1}}

%%% First, define the fields in the memo preamble (i.e, To, From, etc.).
%%% present is \From.
%%%
%%%  \Date - there are two possible places for the date to be printed;
%%%         if this macro is used, a Date: field is output.  If not,
%%%         the string passed to the \memo macro is placed above the
%%%         memo preamble.
%%%
%%%  \From - the author of the memo.
%%%
%%%  \To   - the recipient(s)
%%%
%%%  \Subject - the memo subject (the Subject field)
%%%
%%%  \RE - the Re field
%%%
%%%  \ShortSubj - a short subject to be output in the page header, if the
%%%               fancy package option was specified.  If this isn't specified,
%%%               either the Subject or the Re field data will be used.
%%%
%%%  \Cc - a carbon copy list
%%%
%%%  \File - the name of the file which generated this memo. This is output
%%%          in a smaller (typewriter) font
%%%
%%%  \Version - the version of this memo.  This is output in a smaller font
%%%
%%% The macros place their arguments in other macros with the same name,
%%% but prefixed with the word `memo' (e.g., \memoFrom).  You can use
%%% these if you choose to redefine the page headers and footers.
%%%
%%% If for some reason you've defined a field and you wish to delete
%%% it, use the \DeleteMemoField macro.
%%%
%%%             \DeleteMemoField{File}
%%%
%%% It is harmless to delete a field which hasn't had a value assigned
%%% to it. Assigning an empty value to a field will not delete it.
%%%
%%% You can define a field as many times as you'd like.


%%% remove this line if you're NOT using the rcs option, otherwise
%%% you'll get errors
%\putdraft
\Subject{ }
%\RE{The message you last sent me}
%\ShortSubj{ }
\From{ }
\To{ }
%\Cc{ }

%%% This template uses the RCS information for the Date, Version, and
%%% File fields.  You can override them individually by explicitly
%%% defining them.

%\File{ }

%\Version{1.0}
%\Date{\today}

%%% Generate the memo header.  If the \Date macro isn't specified,

%\memo{\today}

%%% If you wish to redefine the page headers and footers, do so here.
\section{Introduction}
%%% The body of the memo.  No text should appear above here.

\section{Calculations}

It should be noted at the outset that the goal is to calculate a ``worst-case'' UV fluence,
which will necessitate a number of simplifying assumptions that will be detailed throughout this
memo.

\subsection{Flux from Bright Earth}

Our first source of UV flux that we consider is that which is from the Sun, backscattered
from the Earth's atmosphere.

Though \chandra does not directly observe the bright Earth, the Earth can be scanned during
maneuvers. Therefore, in order to determine the total UV fluence reflected from the bright
arth which impinged upon the OBFs, it is necessary to determine the total time accumulated over
the lifetime of the mission when the following two conditions are simultaneously satisfied:

\begin{enumerate}
\item The Earth is in the \chandra field of view
\item ACIS is in the focal plane
\end{enumerate}

We assume that condition 1 is satisfied when the angular radius of the Earth is less than the
angular distance between the Earth and the aimpoint. This was determined by querying
the \code{Ska} engineering archive for the MSIDs \code{Dist\_SatEarth} and \code{Point\_EarthCentAng},
corresponding to the distance between the Earth's center and \chandra ($D_E$) and the angle $\phi_E$ between
the Earth and the aimpoint, respectively. The angular radius of the Earth $\theta_E$ is determined
via $\theta_E = \tan^{-1} (R_\oplus/D_E)$. We assume that condition 2 is satisfied when the TSC
position is $\geq$ -25,000. This can be determined using the MSID \code{3TSCPOS} from the \code{Ska}
archive.

The time resolution of the first two MSIDs in the archive is $\Delta{t} = 5$~min, whereas the resolution
of \code{3TSCPOS} is 32.8~seconds, so the time resolution of the calculation is limited to 5-minute intervals.
This likely results in a modest overestimate of the total length of time that the OBFs have been exposed to
the bright Earth, since the Earth will be entering and exiting the field of view during some of these intervals.
This is consistent with our ``worst-case'' approach to this problem. The total time $\Delta{T}$ within which our two
conditions are satisfied is thus:

\begin{eqnarray}
\Delta{T} &=& \displaystyle\sum_i~\Delta{t_i},~{\rm where}~(\phi_{E,i} \leq \theta_{E,i})~{\rm and}~(\code{3TSCPOS}_i > -25000) \\
\nonumber &=& 2.73~{\rm hr}
\end{eqnarray}

Figure \ref{fig:time_accum} shows the accumulation of this time over the duration of the mission. 

We assume that the UV photons are reflected from the HRMA onto the focal plane with 100\% efficiency,
so the effective area is $A_{\rm eff} = 1145$~cm$^2$. The ``plate scale'' $\Delta{S}$ of the ACIS focal plane is
0.0205"/$\mu$m. The fluence of photons which can break H-C and C=C bonds are then, respectively:

\begin{eqnarray}
H_{\rm H-C} &=& I_{\rm H-C}A_{\rm eff}(\Delta{S})^2\Delta{T} = 4.06 \times 10^{12}~{\rm photons~cm^{-2}} \\
H_{\rm C=C} &=& I_{\rm C=C}A_{\rm eff}(\Delta{S})^2\Delta{T} = 3.38 \times 10^{15}~{\rm photons~cm^{-2}}
\end{eqnarray}

\subsection{Flux from Stars}

\subsection{Flux from Other Planets}

\subsection{Computing the }

\end{document}
